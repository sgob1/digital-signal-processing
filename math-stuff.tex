\usepackage{amsmath,amssymb,amsfonts,bm}
\usepackage{amsthm}
\def\Z{\mathbb Z}
\def\R{\mathbb R}
\def\C{\mathbb C}
\def\N{\mathbb N}
\def\Q{\mathbb Q}

\DeclareMathOperator{\sinc}{sinc}
\DeclareMathOperator{\rect}{rect}
\DeclareMathOperator{\area}{area}
\DeclareMathOperator{\power}{power}
\DeclareMathOperator{\energy}{energy}
\DeclareMathOperator{\mval}{mval}


% Theorems
\newtheoremstyle{newlinethm}{2em}{2em}{\slshape}{2pt}{\bfseries}{\newline}{2em}{}
\theoremstyle{newlinethm}
\newtheorem{thm}{Theorem}[section]
\setlength{\tabcolsep}{0.5em} % for the horizontal padding
{\renewcommand{\arraystretch}{1.35}% for the vertical padding

\newtheorem{predicate}{Predicate}[section]
\setlength{\tabcolsep}{0.5em} % for the horizontal padding
{\renewcommand{\arraystretch}{1.35}% for the vertical padding

\newtheorem{rules}{Rule}[section]
\setlength{\tabcolsep}{0.5em} % for the horizontal padding
{\renewcommand{\arraystretch}{1.35}% for the vertical padding

\theoremstyle{theorem}
\newtheorem{lemma}{Lemma}[section]
\setlength{\tabcolsep}{0.5em} % for the horizontal padding
{\renewcommand{\arraystretch}{1.35}% for the vertical padding

\newtheorem{tesi}{Tesi}[section]
\setlength{\tabcolsep}{0.5em} % for the horizontal padding
{\renewcommand{\arraystretch}{1.35}% for the vertical padding

\theoremstyle{definition}
\newtheorem{defin}{Definition}[section]
\setlength{\tabcolsep}{0.5em} % for the horizontal padding
{\renewcommand{\arraystretch}{1.35}% for the vertical padding

\theoremstyle{definition}
\newtheorem{regola}{Regola}[section]
\setlength{\tabcolsep}{0.5em} % for the horizontal padding
{\renewcommand{\arraystretch}{1.35}% for the vertical padding

\theoremstyle{definition}
\newtheorem{identita}{Identità}[section]
\setlength{\tabcolsep}{0.5em} % for the horizontal padding
{\renewcommand{\arraystretch}{1.35}% for the vertical padding

\theoremstyle{definition}
\newtheorem{predicato}{Predicato}[section]
\setlength{\tabcolsep}{0.5em} % for the horizontal padding
{\renewcommand{\arraystretch}{1.35}% for the vertical padding

\renewcommand\proofname{\textup{\emph{Proof.}}} % default is 'Proof.' 
\renewcommand\qedsymbol{$\blacksquare$}

\newcommand\encircle[1]{%
    \hspace*{2pt}\tikz[baseline=(X.base)] 
    \node (X) [draw, shape=circle, inner sep=-2.5pt, minimum size=0pt] {\strut \footnotesize #1};\hspace*{2pt}}
